\subsection{Complex Numbers}

\begin{tcolorbox}[
        colframe=blue!70!black,
        colback=white,
        boxrule=1pt,
        left=-3mm, right=-3mm, top=1mm, bottom=1mm,
        arc=3mm,
    ]
    \begin{enumerate}[noitemsep, topsep=0pt]
        \item Solve
              \(
              x^2 + 1 = 0 \Rightarrow x = \sqrt{-1}
              \Rightarrow
              \mathbb{C} = \{a + ib : a,b \in \mathbb{R}\}.
              \)
        \item \(
              (a+ib) + (x+iy) = (a+x) + i(b+y),
              \)
        \item \(
              (a+ib)(x+iy) = (ax - by) + i(ay + bx),
              \)
        \item \(
              (a+ib)(a-ib) = a^2 + b^2.
              \)
        \item \(
              \frac{a+ib}{x+iy}
              = \frac{(a+ib)(x-iy)}{x^2+y^2}
              = \frac{ax+by}{x^2+y^2} + i\frac{bx-ay}{x^2+y^2}.
              \)
        \item \(
              |z| = \sqrt{a^2+b^2}, \quad z=a+ib,
              \)
        \item \(
              \overline{a+ib} = a-ib.
              \)
    \end{enumerate}
\end{tcolorbox}
\!\!\!\!\!\!\!\!\!\!\!\!
\tsIdea{R 8.1.1 (Euler's formula)}{For $\theta \in \mathbb{R}$,
    \(
    e^{i\theta} = \cos\theta + i\sin\theta \Rightarrow e^{i\pi} = -1
    \)
}
\newline
\tsIdea{Polar form of a complex number}{\(
z = r e^{i\theta}, \quad z \in \mathbb{C}, \quad r > 0 \text{ is the modulus of } z,
\quad \theta \in [0, 2\pi).
\)}
\newline
\tsThe{8.1.2 (Fundamental Theorem of Algebra)}{Any degree $n$ non-constant ($n \ge 1$) polynomial
    \(
    P(z) = \alpha_n z^n + \alpha_{n-1} z^{n-1} + \cdots + \alpha_1 z + \alpha_0,
    \quad (\alpha_n \neq 0)
    \)
    has a zero: there exists $\lambda \in \mathbb{C}$ such that
    \(
    P(\lambda) = 0.
    \)}
\\
\(\Rightarrow\) A degree-$n$ polynomial has at most $n$ distinct zeros (roots).
\newline
\tsCor{8.1.3 (Algebraic multiplicity, num. of 0 in polynomial)}{Any degree $n$ non-constant ($n \ge 1$) polynomial has $n$ zeros
    \(
    \lambda_1, \ldots, \lambda_n \in \mathbb{C},
    \)
    and
    \(
    P(z) = \alpha_n (z - \lambda_1)(z - \lambda_2)\cdots(z - \lambda_n).
    \)
    The number of times $\lambda \in \mathbb{C}$ appears in the expression is called the
    \emph{algebraic multiplicity} of the zero.}
\newline
\tsIdea{Inner product on $\mathbb{C}^n$}{The inner product on $\mathbb{C}^n$ is given by
    \(
    \langle v, w \rangle = w^* v.
    \)
}
\newline
\tsIdea{Conjugate transpose}{\(
    A^* = \overline{A}^{\,T}.
    \)
}