\subsection{Diagonalization}
\tsThe{9.1.1 (Diagonalization Theorem, ability changing basis)}{$A = V \Lambda V^{-1}$, where $V$’s columns are its eigenvectors and $\Lambda$ is a diagonal matrix with $\Lambda_{ii} = \lambda_i$ and all
    other entries $0$. $A \in \mathbb{R}^{n \times n}$ and has to have a complet set of real eigenvectors (eigenbasis).
    \\ Equivalently,
    \(
    \Lambda = V^{-1} A V,
    \)
    since $V$ is invertible.
}
\[
    \text{Std. coord.}
    \;\xrightarrow{V^{-1}}\;
    \text{EV. coord.}
    \;\xrightarrow{\Lambda}\;
    \text{EV. coord.}
    \;\xrightarrow{V}\;
    \text{Std. coord.}
\]
\tsDef{9.1.2 (Diagonalizable matrix)}{A matrix $A \in \mathbb{R}^{n \times n}$ is called \emph{diagonalizable} if there exists
    an invertible matrix $V$ such that
    \(
    V^{-1} A V = \Lambda,
    \)
    where $\Lambda$ is a diagonal matrix.
}
\newline
\tsDef{9.1.3 (Complete set of EV)}{If we can find eigenvectors forming a basis of $\mathbb{R}^n$ for $A$, we say that
    $A$ has a \emph{complete set of real eigenvectors}.
}
\newline
\tsProp{9.1.6 (Projection and EW/EV)}{Let $P$ be a projection matrix onto a subspace $U \subset \mathbb{R}^n$.
    Then $P$ has two eigenvalues, $0$ and $1$, and a complete set of real
    eigenvectors.
}
\newline
\tsDef{9.1.7 (Similar matrices)}{Matrices $A \in \mathbb{R}^{n \times n}$ and $B \in \mathbb{R}^{n \times n}$
    are called \emph{similar} if there exists an invertible matrix $S$ such that
    \(
    B = S^{-1} A S.
    \)
    \textbf{P 9.1.8}: Similar matrices have the same eigenvalues.
}
\newline
\tsDef{9.1.10 (Geometric multiplicity)}{Let $A \in \mathbb{R}^{n \times n}$ and let $\lambda$ be an eigenvalue of $A$.
    Then
    \(
    \dim \mathcal{N}(A - \lambda I)
    \)
    is called the \emph{geometric multiplicity} of $\lambda$.
}
\newline
\tsLem{9.1.11 (Complete set of real EV)}{A matrix has a complete set of real eigenvectors if and only if all its
    eigenvalues are real and the geometric multiplicities equal the algebraic
    multiplicities for all eigenvalues.
}
